% !TeX root = ../main.tex

\section{Clustering}

\paragraph{Variants of k-means:}
For clustering see section 14 of ``The Elements of Statistical Learning'' (Hastie, Tibshirani, Friedman - 2009). Section 14.3 ``Cluster Analysis'' introduces different flavours of k-means. Points of interest are:
\begin{itemize}
	\item Proximity Matrices
	\item K-means
	\item Gaussian Mixtures as Soft K-means Clustering
	\item Vector Quantization
\end{itemize}

\paragraph{Q:} How can we determine k?

Let $w(c)$ be distance from samples to cluster centres within clusters. We use this as a metric of quality in cluster analysis.

\subparagraph{Approach 1:} Track the rate of change of a quality metric (like $w(c)$). Proposed in ``Pattern Classification'' (Duda, Hart, Stork).

\begin{figure}[H]
	\centering
	\includegraphics[width=0.8\textwidth]{07-rate-of-change}
\end{figure}

\subparagraph{Approach 2:} Let $w'(c)$ be a metric on a uniform distribution of samples. Relate change of $w(c)$ to change of $w'(c)$. Proposed in TEoSL.

\begin{figure}[H]
	\centering
	\includegraphics[width=0.8\textwidth]{08-w_c-vs-w_prime_c}
\end{figure}