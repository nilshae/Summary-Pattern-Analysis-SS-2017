% !TeX root = ../main.tex

\section*{Manifold Learning}
\paragraph{Principal idea:} Reduce the dimensions of the Data, but preserve the structure of the data / underling manifold.

\subsection*{Curse of Dimensionality}
``Human intuition breaks in high dimensional spaces'' - (Bellman, 1961)

\paragraph{Let's illustrate this:}
Consider a d-dimensional feature vector $\vec{x_1}, \vec{x_2}, \dots, \vec{x_N} \in \mathbb{R}^d$ where $0 \le x_{i,k} \le 1$ (??? wo kommt das k her) uniformly distributed in a d-dimensional hyperspace of volume 1.
Let's say we would like to cover enough volume of  the cube to collect $1\%$ of the data. Let's say we also use a cube for this task. What is the required edge length $s$ of the cube to obtain that $1\%$ of space?

\paragraph{Example:}
A 10 dimensional hypercube

\begin{equation*}
    V=s^d \Rightarrow s = V^{\frac{1}{d}} = 0,01^{\frac{1}{10}} = 0.63
\end{equation*}

Another way of thinking about this is, that in a very high dimensional space, virtually every feature point is located at the boundary of the feature space\footnote{Because in at least one dimension, we draw a very low of very high value}. This leads to the effect, that common distance measures loose their  effectivity. E.g. the \textit{median} distance for the nearest neighbour to the origin.

\paragraph{Example applications:}
(``Notorious'' examples for high dimensional data)

\begin{itemize}
    \item hyper-spectral remote sensing image classification
    \item satellite image:  perform e.g. agricultural monitoring classify type of vegetation from hyper-spectral (many color channels) image.
\end{itemize}

In such classification pipelines, dimensionality reduction is often one integral component.
