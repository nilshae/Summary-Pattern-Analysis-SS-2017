% !TeX root = ../main.tex

\section{Manifold Learning}
Principal idea: Reduce the dimensions of the Data, gut preserve the structure of the data / underling manifold.

\subsection{Curse of Dimensionality}
``Human intuition breaks in high dimensional spaces :(''

\paragraph{Let's illustrate this:}
Consider a d-dimensional feature vector $\vec{x_1}, \vec{x_2}, \dots, \vec{x_N} \in \mathbb{R}^d$ where $0 \le x_{i,k} \le 1$ (??? wo kommt das k her) uniformly distributed in a d-dimensional hyperspace of volume 1.
Let's say we would like to cover enough volume of  the cube to collect $1\%$ of the data. Let's say we also use a cube for this task. What is the required edge length $s$ of the cube to obtain that $1\%$ of space?

\paragraph{Example:}
A 10 dimensional hypercube

\begin{equation*}
    V=s^d \Rightarrow s = V^{\frac{1}{d}} = 0,01^{\frac{1}{10}} = 0.63
\end{equation*}

\begin{figure}[H]
	\centering
	\includegraphics[width=0.5\textwidth]{todo}
\end{figure}

Another way of thinking about this is, that in a very high dimensional space, virtually every feature point is located at the boundary of the feature space\footnote{Because in at least on dimension, we draw a very low of very high value}. This leads to the effect, that common distance measures loose their  effectivity. E.g. the \textit{median} distance for the nearest neighbour to the origin.

\subsection{Techniques for dimensionality reduction}
\paragraph{Recap:}
``Notorious'' example for high dimensional data:
\begin{itemize}
    \item hyper-spectral remote sensing image classification
    \item satellite image:  perform e.g. agricultural monitoring classify type of vegetation from hyper-spectral (many color channels) image.
\end{itemize}

Our known approch is PCA.

\begin{figure}[H]
	\centering
	\includegraphics[width=0.5\textwidth]{todo}
\end{figure}

Orthogonal basis that is aligned with the ``maximum spread'' (w.r.t. the covariance) of the data. It is a global unsupervised method\footnote{Operating on all data points, no labels, find one global optimal solution}. Consider that PCA os a linear method.

\begin{figure}[H]
	\centering
	\includegraphics[width=0.5\textwidth]{todo}
\end{figure}

The Kernel PCA performs a non-linear mapping of the data, then perform a standard (linear) PCA on the result. With the kernel-trick we can do that in one step. The Objective function: (the non-linear mapping is part of $\phi$)

\begin{equation*}
    \delta = sum_{i,j=1}^{N} (\phi(\vec{x_i}) - \phi(\vec{x_j}))^T (\phi(\vec{x_i}) - \phi(\vec{x_j})) + \lambda (\phi^T\phi-1)
\end{equation*}

Some offsprings of the Kernel PCA idea are:
\begin{enumerate}
    \item Perform some preprocessing / mapping that operates non-linearly
    \item The dimensionality reduction itself is an operation that operates linearly.
\end{enumerate}

\subsection{Multi Dimensional Scaling (MDS)}
TODO:

\subsection{ISOMAP Algorithm}
A non-linearity ``patch'' to MDS

\begin{figure}[H]
	\centering
	\includegraphics[width=0.5\textwidth]{todo}
\end{figure}

\paragraph{Idea:} Nearby points have their ``usual'' euclidian distance. If a pair of points are not within a local neigbourhood, then the distance between these points is a graph distance\footnote{Compute the all paths shortest path (Dejkstra, A*,  DFS, Floyd–Warshall, ...)}. Then run MDS on the resulting distance Matrix.

\subsection{Locally Linear Embedding (LLE)}
TODO:

\subsection{Laplacian Eigenmaps}
TODO:
