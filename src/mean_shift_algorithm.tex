% !TeX root = ../main.tex

\section{Mean Shift Algorithm}

\paragraph{Purpose:}
Find maximum in pdf without actually performing a full density estimation.

\paragraph{Potential applications:}
Clustering, segmentation, \ldots{} \\

Assume that we have a full density estimator.

\begin{equation*}
  p(\vec{x}) = \dfrac{1}{N} \sum_{i=1}^N k(\vec{x}; \vec{x_i})
\end{equation*}

Idea: Maxima can be found, where the gradient of the pdf is zero.

\begin{figure}[H]
  \centering
  \includegraphics[width=0.3\textwidth]{todo}
  \caption{The kernel size indirectly controls the number of indentified maxima}
\end{figure}
% use subfigure
\begin{figure}[h]
  \centering
  \includegraphics[width=0.3\textwidth]{todo}
  \caption{One of the issues is, the case when a zero gradient is just between two finer maxima}
\end{figure}
