% !TeX root = ../main.tex

\subsection*{Laplacian Eigenmaps}

% \textbf{Graph Laplacian:} Matrix representation of a graph. There are many other ways to do this.
%
% \[L = (D - W), \]
% where \(D\) is a diagonal matrix with entries \[d_{ij} = \begin{cases} \sum_{k=1}^N w_{ik} & if \ i=j \\ 0 & if \ i \neq j \end{cases}\] and \(W\) is the adjacency matrix \[w_{ij} = \begin{cases} \text{weight between nodes i and j} & if \ i\neq j \\ 0 & if \ i = j\end{cases}\]
%
% We exclude self-loops here. D is there for normalization. This results in a matrix, which row sum equals \(0\) (\(\sum_{k=1}^N l_{ik} = 0\)).
%
% The graph Laplacian is a popular tool to operate on graphs with methods from linear algebra (\(\rightarrow\) this is a topic in "spectral graph theory").

\paragraph{Idea:}
\begin{enumerate}
    \item Build an adjacency graph on the set of feature points \(S=\{x_1, \dots, x_n\}\)
    \item Choose weights \(w_{ij}\) for the edges of the graph
    \item Perform eigendecomposition of the graph Laplacian
    \item Obtain low-dimensional embedding
\end{enumerate}
Choose the weights as affinities, meaning that closer point pairs have higher weights \(\rightarrow\) points that are closely together in high-dimensional space shall remain close in the lower-dimensional embedding.

One common choice for the weights is the so-called heat kernel \(w_{ij} = e^{-||x_i - x_j||_2^2}\), or just binary affinity
$w_{ij} = 1$ if $ ||x-x_j||_2^2 \leq t $ and  $0$ otherwise.

\bigbreak

\paragraph{Ojective function:}
\[minimize \sum_{i=1}^{N}\sum_{j=1}^{N}||x_i'- x_j'||_2^2 w_{ij} \qquad \text{where} \quad x'\in \mathbb{R}^{d'}, d' \ll d \]

Constraint for optimization: \(x'^T D x' = 1\) (squared distances should add up to 1).

Relationship of the objective function to the graph Laplacian:
\begin{align*}
    \sum_{i,j} ||x_i' - x_j'||_2^2 w_{ij} &= \sum_{i,j} (x_i'^T x_i' + x_j'^T x_j' - 2 x_i'^T x_j') w_{ij} = (2\cdot \sum_{i,j=1}^N(x_i'^T x_i') \cdot w_{ij}) - (2 \cdot \sum_{i,j} x_i'^T x_j' w_{ij})\\
    &= (2 \cdot \sum_{i=1}^N x_i'^T x_i' \cdot (\sum_{j=1}^N w_{ij}))- (2 \cdot \sum_{i,j} x_i'^T x_j' w_{ij})\\
    &= 2 x'^T D x' - x'^TWx' = 2 \cdot x'^T (D-W) x'
\end{align*}

\[\text{minimize } x'^T L x' \text{ subject to } x'^T D x' = 1:\]
\begin{align*}
    &\frac{\partial}{\partial x'} (x'^T L x + \lambda (1-x'^T D x'))\\
    &= 2 L x' - \lambda D x' = 0\\
    &= 2 L x' - \lambda D x' = \vec{0}\\
    Lx' &= \lambda D x'\\
    \Rightarrow D^{-1} L x' &= \lambda x'
\end{align*}

On 4.) The lower-dimensional embedding is obtained by selecting a ordered subset of eigenvectors from that decomposition.
\begin{itemize}
    \item
        Sort eigenvectors by the magnitude of their associated eigenvalues
    \item
        Discard the eigenvalue/eigenvector pair for the lowest eigenvalues (the 0 eigenvalue) (number of values depends on connected components in graph)
    \item
        The \(d'\) next smallest eigenvalue/eigenvector pairs (\(e_1, e_2, \dots\)) form the lower -dimensionanl embedding: \(i\)th row of \((e_1, e_2, ... e_d)\) represents the lower dimensioanl embedding of \(x_i^T\).
\end{itemize}

\textit{Addendum} to our Section on clustering: "Spectral clustering" is essentially executing Laplacian Eigenmaps + \(k\)-Means on the lower dimensional space.

How to choose a appropriate \(d'\)? The distribution of the eigenvalues gives an indication about the "intrinsic" dimensionality of the data. If there is a certain gap between eigenvalue, this is a good indication. We don't have this available when using LLE.
